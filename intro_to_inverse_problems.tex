\documentclass[12pt]{amsart}
\RequirePackage[l2tabu, orthodox]{nag}
\usepackage[main=english]{babel}
\usepackage[rm={lining,tabular},sf={lining,tabular},tt={lining,tabular,monowidth}]{cfr-lm}
\usepackage{amsthm,amssymb,latexsym,gensymb,mathtools,mathrsfs}
\usepackage[T1]{fontenc}
\usepackage[utf8]{inputenc}
\usepackage[pdftex]{graphicx}
\usepackage{epstopdf,enumitem,microtype,dcolumn,booktabs,hyperref,url,fancyhdr}
\usepackage[margin=2.7182818284590452353602874713526624977572470936999cm]{geometry}
\pagestyle{fancy}
\setlength{\headheight}{15pt}

% Plotting
\usepackage{pgfplots}
\usepackage{xinttools} % for the \xintFor***
\usepgfplotslibrary{fillbetween}
\pgfplotsset{compat=1.8}
\usepackage{tikz}

% Custom Commands
\newcommand*{\norm}[1]{\left\lVert#1\right\rVert}
\newcommand*{\abs}[1]{\left\lvert#1\right\rvert}
\newcommand*{\suchthat}{\,\mathrel{\big|}\,}
\newcommand{\E}{\mathbb{E}}
\newcommand{\Var}{\mathrm{Var}}
\newcommand{\Img}{\mathrm{Img}}
\newcommand{\Ker}{\mathrm{Ker}}
\newcommand{\Cov}{\mathrm{Cov}}
\newcommand{\Prob}{\mathbb{P}}
\DeclarePairedDelimiterX\innerp[2]{(}{)}{#1\delimsize\vert\mathopen{}#2}
\DeclareMathOperator*{\argmax}{argmax}
\DeclareMathOperator*{\argmin}{argmin}

\setlist{topsep=1ex,parsep=1ex,itemsep=0ex}
\setlist[1]{leftmargin=\parindent}
\setlist[enumerate,1]{label=\arabic*.,ref=\arabic*}
\setlist[enumerate,2]{label=(\alph*),ref=(\alph*)}
\begin{document}

\section{Introduction} 
The study of inverse problems is generally concerned with approximating an unknown value using potentially corrupted or noisy data. The study of such problems finds many applications across the physical, social, and mathematical sciences. 
\textbf{TODO}
\begin{enumerate} 
\item Brief history of field
\item Purpose/overview of paper
\item Organization of paper
\end{enumerate} 

\section{Inverse Problems}

\subsection{Introduction} \hfill \\
Students of linear algebra may recall the prominence of the matrix equation $\textbf{Ax} = \textbf{b}$. Here we may consider $\textbf{A}$ as a linear map that acts on a given vector $\textbf{x}$ and produces the output vector $\textbf{b}$. This constitutes the ``forward problem'' in which we are given inputs and must calculate outputs. The corresponding ``inverse problem'' reverses the situation; we observe $\textbf{b}$ and must determine the input $\textbf{x}$. In its simplest form we may assume $\textbf{A}$ is invertible, in which case the solution to the inverse problem simply involves matrix inversion: $x = A^{-1}b$. This paper introduces a vast generalization of this inverse problem framework, shedding the simplifying assumptions of finite dimensionality, invertibility, and existence of a solution. The motivation for such a general theory of inverse problems will become clear with examples, as the majority of interesting applications fail to enjoy the ``well-posedness'' of the above equation. To make these notions concrete, consider the following equation:

\begin{align*} 
&F: X \to Y && F(x) = y && (1.1)
\end{align*} 

This equation, which generalizes the above matrix equation, will constitute the primary subject of our analysis. The goal remains to compute $x$ given $y$. For the time being, I have been intentionally vague regarding specific properties of the map $F$ or spaces $X$ and $Y$. The ensuing sections will demonstrate that different assumptions about these objects will result in different avenues for analysis and produce different results. 

\subsection{Ill-Posed Problems} \hfill \\
The following definition highlights the central difficulties in studying inverse problems, and motivates much of the theory discussed in this paper. \\


\textbf{Definition} \textit{Hadamard's Conditions for Well-Posedness}: (EL) Problem (1.1) is \textit{well-posed} in the sense of Hadamard provided it meets the three following conditions: 
\begin{enumerate} 
\item (Existence) A solution always exists; that is, given any $y \in Y$ there exists an $x \in X$ such that $F(x) = y$. 
\item (Uniqueness) The solution is unique; that is, if $F(x_1) = F(x_2)$ then $x_1 = x_2$.
\item (Stability) The solution is a continuous function of the input data; that is, if $F(x_n) \to y$  then $x_n \to x$ \\
If any of these conditions are not met, the problem is called \textit{ill-posed}. 
\end{enumerate} 

The treatment of of ill-posedness is the primary focus in the study of inverse problems, in particular the development of techniques to handle violations of the third condition. Violations of the first two conditions are typically handled by instead seeking the ``closest'' solution in some sense; that is, by reformulating the problem as a minimum norm problem: 

\begin{align*}
&\argmin_{x \in X} \norm{F(x) - y}
\end{align*}
 
 The violation of stability presents a more difficult challenge. In general we imagine $y$ to be data measured imprecisely. Therefore, the concern is that small fluctuations in the observations $y$ may lead to large deviations in $x$. Regularization methods form the foundation of the toolkit used to address this problem and will be discussed later in this paper. 
 
 \subsection{Theoretical Framework} \hfill \\
 
 \subsection{Examples} \hfill \\
 An abstract, general formulation of inverse problems may not initially appear useful. However, it soon becomes clear that the generality allows these techniques to be applied to a wide array of problems across various fields. Many of these applications are naturally formulated in infinite-dimensional vector spaces, and moreover suffer from ill-posedness. I provide three examples below of such applications. 
 
 \subsubsection{Statistical Estimation} \hfill \\
 Parameter estimation (or \textit{model calibration}), ubiquitous across the fields of statistics and machine learning, may be formulated as an inverse problem. In this case we view $F_\theta$ as a model or process that depends on some parameters $\theta$. We may alternatively view $\theta$ as an additional input to the system, in which case $F$ acts on both the parameters and input data to produce the observed outputs; that is, $F(\theta, x) = y$. The problem thus becomes: given observed outputs $y$ and inputs $x$, estimate the parameters $\theta$ of the underlying process $F$. 
 
 \section{Preliminaries: Operator Theory and Probability}
 This paper assumes basic knowledge of the major results in the study of Banach and Hilbert Spaces, including the basic theory of linear maps between such spaces. However, 
 
 
 However, no background in operator theory is assumed. Given how intertwined the study of inverse problems is with the study of operators, all of the needed results are detailed here. Proofs are provided when deemed useful, with some being relegated to the appendix. 
 
 \subsection{Operator Theory} \hfill \\
 Throughout this paper, all vector spaces are assumed to be real. 
 
 \textbf{Definition}: A \textit{map} $F: X \to Y$ is function mapping between two vector spaces $X$ and $Y$, called the \textit{domain} and \textit{codomain}, respectively. If $F$ maps $x \in X$ to $y \in Y$ I write $F(x) = y$. I denote by $\Img F$ the set of all $y \in Y$ such that there is an $x \in X$ with $F(x) = y$. I denote by $\Ker F$ the set of all $x \in X$ that map to the zero vector $0 \in Y$. Letting $U \subset Y$, I define the \textit{inverse image} (or \textit{pre-image}) $F^{-1}(V)$ to be the set of all $x \in X$ satisfying $F(x) \in V$. 
 
 Throughout this paper, we restrict our attention to the study of linear inverse problems. Therefore, the following definition is central to the theory. 
 
 \textbf{Definition}: A \textit{linear map} (or \textit{operator}) is a map $F: X \to Y$ satisfying $F(\alpha_1x_1 + \alpha_2x_2) = \alpha_1F(x_1) + \alpha_2F(x_2)$ for all real numbers $\alpha_1, \alpha_2$ and vectors $x_1, x_2 \in X$. When a map is linear, I suppress the parentheses and write function evaluation as $Fx := F(x)$. 

In a manner similar to linear functional theory, it can be shown that the space of linear maps from $X$ to $Y$ is a vector space. Moreover, if $X$ and $Y$ are normed spaces we may define a norm on this space in similar fashion. 

\textbf{Definition}: Let $F$ be a linear map between normed linear spaces $X$ and $Y$. Define the norm of $F$ $\norm{F} := \sup_{x \neq \theta} \frac{\norm{Fx}}{x}$. 

It can, of course, be shown that this does indeed define a norm on this space. 


\section{Linear Inverse Problems in Hilbert Space}

Hilbert space provides a natural setting in which to study inverse problems. From a theoretical point of view, the rich theory of Hilbert space leads to relatively simple characterizations of solutions, while practically, many real-world inverse problems may naturally be formulated in such settings. \textbf{Insert example here that uses sum of squares norm}. Throughout this section, $X$ and $Y$ are assumed to be real Hilbert spaces with respective inner products $(\cdot | \cdot)_X$ and $(\cdot | \cdot)_Y$. Recall from section \textbf{2} that the natural approach to address the non-existence and non-uniqueness of $Tx = y$ was to consider the least-squares problem $\inf_{x \in X} \norm{Tx - y}_Y$. If we assume that $\Img T \subset Y$ is closed, this guarantees the existence of a unique $\hat{y} \in Y$ minimizing $\norm{y^\prime - y}_Y$ on $\Img T$. However, we have not assumed that $T$ is injective and hence this does not imply the uniqueness of a least-squares solution. Therefore, we must study the set of all least-squares solutions, with an emphasis on the solution of minimal norm. 

 \textbf{Definition} (generalized solutions): 
 \begin{itemize}
 \item $x \in X$ is called a \textit{least-squares solution} of 1.1 if $\norm{Tx - y}_Y \leq \norm{Tu - y}_Y$ for all $u \in X$. $\mathbb{L}$ denotes the set of all such solutions. 
 \item $x^\dagger \in X$ is called a \textit{minimal norm least-squares solution} if $x^\dagger \in \mathbb{L}$ and $\norm{x^\dagger}_X \leq \norm{x}_X$ for all $x \in \mathbb{L}$. 
 \end{itemize} 
 
 Note that in general $\mathbb{L}$ may be empty. The following assumption relies on the assumption that $\Img T$ is closed to ensure the existence of a solution. 
 
 \textbf{Proposition}: Suppose $\Img T$ is closed. Then there is a unique minimal norm least-squares solution $x^\dagger$. Moreover, $x^\dagger \in (\Ker T)^\perp$. \\[.1cm]
 
 \textit{Proof}: Since $\Img T$ is closed, then there exists a unique $\hat{y} \in Y$ satisfying $\hat{y} = \min_{y^\prime \in \Img T} \norm{y^\prime - y}_{Y}$. Let $\hat{x} \in \mathbb{L}$ such that $T\hat{x} = \hat{y}$. Then, 
 \begin{align*} 
 \mathbb{L} &= \{x \in X: Tx = \hat{y}\} \\
                   &= \{x \in X: T(x - \hat{x}) = 0\} \\
                   &= \hat{x} + \{x - \hat{x} \in X: T(x - \hat{x}) = 0\} \\
                   &= \hat{x} + \Ker T
 \end{align*}  
 
 Therefore, $ \mathbb{L}$ is a translation of the closed subspace $\Ker T$. It then follows immediately from Luenberger 3.10 Theorem 3 that there exists a unique $x^\dagger \in \mathbb{L}$ of minimal norm, and moreover that $x^\dagger$ is orthogonal to $\Ker T$. $\qquad \blacksquare$

\end{document} 