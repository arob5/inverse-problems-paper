\documentclass[12pt]{amsart}
\RequirePackage[l2tabu, orthodox]{nag}
\usepackage[main=english]{babel}
\usepackage[rm={lining,tabular},sf={lining,tabular},tt={lining,tabular,monowidth}]{cfr-lm}
\usepackage{amsthm,amssymb,latexsym,gensymb,mathtools,mathrsfs}
\usepackage[T1]{fontenc}
\usepackage[utf8]{inputenc}
\usepackage[pdftex]{graphicx}
\usepackage{epstopdf,enumitem,microtype,dcolumn,booktabs,hyperref,url,fancyhdr}
\usepackage[margin=2.7182818284590452353602874713526624977572470936999cm]{geometry}
\pagestyle{fancy}
\setlength{\headheight}{15pt}

% Plotting
\usepackage{pgfplots}
\usepackage{xinttools} % for the \xintFor***
\usepgfplotslibrary{fillbetween}
\pgfplotsset{compat=1.8}
\usepackage{tikz}

% Custom Commands
\newcommand*{\norm}[1]{\left\lVert#1\right\rVert}
\newcommand*{\abs}[1]{\left\lvert#1\right\rvert}
\newcommand*{\suchthat}{\,\mathrel{\big|}\,}
\newcommand{\E}{\mathbb{E}}
\newcommand{\Var}{\mathrm{Var}}
\newcommand{\Cov}{\mathrm{Cov}}
\newcommand{\Prob}{\mathbb{P}}
\DeclarePairedDelimiterX\innerp[2]{(}{)}{#1\delimsize\vert\mathopen{}#2}
\DeclareMathOperator*{\argmax}{argmax}
\DeclareMathOperator*{\argmin}{argmin}

\setlist{topsep=1ex,parsep=1ex,itemsep=0ex}
\setlist[1]{leftmargin=\parindent}
\setlist[enumerate,1]{label=\arabic*.,ref=\arabic*}
\setlist[enumerate,2]{label=(\alph*),ref=(\alph*)}
\begin{document}

\section{Introduction} 
The study of inverse problems is generally concerned with approximating an unknown value using potentially corrupted or noisy data. The study of such problems finds many applications across the physical, social, and mathematical sciences. 
\textbf{TODO}
\begin{enumerate} 
\item Brief history of field
\item Purpose/overview of paper
\item Organization of paper
\end{enumerate} 

\section{Inverse Problems}
Students of linear algebra may recall the prominence of the matrix equation $\textbf{Ax} = \textbf{b}$. Here we may consider $\textbf{A}$ as a linear map that acts on a given vector $\textbf{x}$ and produces the output vector $\textbf{b}$. This constitutes the ``forward problem'' in which we are given inputs and must calculate outputs. The corresponding ``inverse problem'' reverses the situation; we observe $\textbf{b}$ and must determine the input $\textbf{x}$. In its simplest form we may assume $\textbf{A}$ is invertible, in which case the solution to the inverse problem simply involves matrix inversion: $x = A^{-1}b$. This paper introduces a vast generalization of this inverse problem framework, shedding the simplifying assumptions of finite dimensionality, invertibility, and existence of a solution. The motivation for such a general theory of inverse problems will become clear with examples, as the majority of interesting applications fail to enjoy the ``well-posedness'' of the above equation. To make these notions concrete, consider the following equation:

\begin{align*} 
&F: X \to Y && F(x) = y && (1.1)
\end{align*} 

This equation, which generalizes the above matrix equation, will constitute the primary subject of our analysis. The goal remains to compute $x$ given $y$. For the time being, I have been intentionally vague regarding specific properties of the map $F$ or spaces $X$ and $Y$. The ensuing sections will demonstrate that different assumptions about these objects will result in different avenues for analysis and produce different results. The following definition highlights the central difficulties in studying this problem, and motivates much of the theory discussed in this paper. \\


\textbf{Definition} \textit{Hadamard's Conditions for Well-Posedness}: (EL) Problem (1.1) is \textit{well-posed} in the sense of Hadamard provided it meets the three following conditions: 
\begin{enumerate} 
\item (Existence) A solution always exists; that is, given any $y \in Y$ there exists an $x \in X$ such that $F(x) = y$. 
\item (Uniqueness) The solution is unique; that is, if $F(x_1) = F(x_2)$ then $x_1 = x_2$.
\item (Stability) The solution is a continuous function of the input data; that is, if $F(x_n) \to y$  then $x_n \to x$ \\
If any of these conditions are not met, the problem is called \textit{ill-posed}. 
\end{enumerate} 

The treatment of of ill-posedness is the primary focus in the study of inverse problems, in particular the development of techniques to handle violations of the third condition. Violations of the first two conditions are typically handled by instead seeking the ``closest'' solution in some sense; that is, by reformulating the problem as a minimum norm problem: 

\begin{align*}
&\argmin_{x \in X} \norm{F(x) - y}
\end{align*}
 
 The violation of stability presents a more difficult problem. In general we imagine $y$ to be data measured imprecisely. Therefore, the concern is that small fluctuations in the observations $y$ lead to large deviations in $x$. 



\end{document} 